\documentclass[11pt]{beamer}
\usepackage[utf8]{inputenc}
\usepackage[T1]{fontenc}
\usepackage{lmodern}
\usepackage[portuguese]{babel}
\usetheme{default}

\setbeamertemplate{navigation symbols}{}
\begin{document}
	\author{Ozéias Souza}
	\title{Algoritmos Genéticos}
	%\subtitle{}
	%\logo{}
	%\institute{}
	%\date{}
	%\subject{}
	%\setbeamercovered{transparent}
	%\setbeamertemplate{navigation symbols}{}
	\begin{frame}[plain]
		\maketitle
	\end{frame}
	
 	% Introdução algoritmo genetico
	\begin{frame}
		\frametitle{Algoritmos Genéticos}
		\begin{itemize}
			\setlength\itemsep{1em}
			\item Algoritmos Genéticos são inspirados no princípio Darwiniano da evolução das espécies e na genética.
			
			\item São algoritmos probabilísticos que fornecem um mecanismo de busca paralela e adaptativa baseado no princípio de sobrevivência dos mais aptos e na reprodução.
		\end{itemize}
	\end{frame}

	% Principios
	\begin{frame}
		\frametitle{Algoritmos Genéticos: Princípios}
		\begin{itemize}
			\setlength\itemsep{1em}
			\item Algoritmos Genéticos constituem uma técnica de busca e otimização inspirada no princípio Darwiniano de seleção natural e reprodução genética.
			
			\item O principio da seleção natural privilegia os indivíduos mais aptos e, portanto, com maior taxa de reprodução.
			
			\item Indivíduos com uma maior probabilidade de reprodução tem mais chances de perpetuarem seus códigos genéticos.
			
		\end{itemize}
	\end{frame}

	% Principios
	\begin{frame}
		\frametitle{Algoritmos Genéticos: Aplicação}
		\begin{itemize}
			\setlength\itemsep{1em}
			\item Estes princípios são imitados na
			construção de algoritmos computacionais que buscam uma melhor solução para um determinado problema.
			
			\item Em AGs um cromossoma é uma estrutura de dados que representa uma das possíveis soluções.
			
			\item Cromossomas são então submetidos a um processo evolucionário que envolve avaliação, seleção, recombinação (crossover) e mutação.
												
		\end{itemize}
	\end{frame}

	% Caracterização do algoritmo
	\begin{frame}
		\frametitle{Algoritmos Genéticos: Características de um AG}
		\begin{enumerate}
			\item \textbf{Problema de otimização}: Problemas com diversos parâmetros ou características que precisam ser combinadas para se obter melhores soluções.
			\item  \textbf{Representação das soluções}: Definindo a estrutura do cromossomo a ser manipulado.
			\item  \textbf{Decodificação do cromossomo}:Consiste na construção da solução do problema a partir do cromossomo.
			\item \textbf{Avaliação}: Realizada a partir de uma função que melhor representa o problema.
			\item \textbf{Seleção}: Seleciona indivíduos para reprodução
			\item  \textbf{Operadores genéticos}: Indivíduos selecionados são recombinados através do crossover.
			\item  \textbf{Inicialização da população}: A população inicial é
			formada a partir de indivíduos aleatoriamente criados.
		\end{enumerate}
	\end{frame}

\end{document}